\documentclass[12pt]{article}
\usepackage{amsmath}
\usepackage{amsfonts}
%opening
\title{Flat Norm On Graphs Rewrite}
\author{Curtis Michels}

\begin{document}

\maketitle

\begin{abstract}
Here we have some derivations towards computing the flat norm on irregularly spaced graphs in $\mathbb{R}^d$.
\end{abstract}

\section{Notation and definitions}

Let $V = \{v_1,v_2,...,v_N\}$ be a set of vertices with $v_i \in \mathbb{R}^d: i = 1,...,N$ and let $E \subseteq V^2$ be a set of symmetric edges on $V$.\\

Let $v_k \in V$ and denote its degree by $D$. Define for all vertices $\{v_i\}_{i=1}^D$ connected to $v_k$ the edge vector $u_i := v_i-v_k$. We associate with each $u_i$ a variable weight $\omega_i \in \mathbb{R}$. Then for any $\nu$ in the $d$-dimensional unit sphere $\partial B(0,1)$ we define the functional:

\begin{align*}
	F(\nu,\omega_1,...,\omega_D) &:= \sum_{i=1}^D \omega_i |\langle \nu,u_i \rangle|\\
	&\equiv \sum_{i=1}^D \omega_i u_i(\nu)\\
\end{align*}

\section{Calculation}

From this we construct our functional of interest:

\begin{align*}
	\mathcal{F}(\omega_1,...,\omega_D) &:= \int_{\partial B(0,1)} |F(\nu,\omega_1,...,\omega_D) - \|\nu\||^2 d \nu\\
	&= \int_{\partial B(0,1)} |F(\nu,\omega_1,...,\omega_D) - 1|^2 d \nu\\
\end{align*}

Taking $|a-b|^2 = (a-b)^2$ for any real $a$ and $b$, we expand:

\begin{align*}
	\mathcal{F}(\omega_1,...,\omega_D) &= \int_{\partial B(0,1)} \left(\sum_{i=1}^D \omega_i u_i(\nu) - 1\right)^2d\nu\\
	&= \int_{\partial B(0,1)} \sum_{i=1}^D \omega_i^2 u_i(\nu)^2 + 2 \sum_{i=1}^D \sum_{j=1}^{i-1}\omega_i \omega_j  u_i(\nu) u_j(\nu) - 2\sum_{i=1}^D \omega_i u_i(\nu) + 1d\nu
\end{align*}

Which makes use of the formula $\left[\sum_{i=1}^m a_i\right]^2 = \sum_{i=1}^m a_i^2 + 2\sum_{i=1}^m \sum_{n=1}^{m-1}a_i a_n$.\\

Define $\omega_{ij} := \omega_i \omega_j $ and $u_{ij}:=u_iu_j$. Then we have:

\begin{align*}
	&= \sum_{i=1}^D \omega_i^2 \int_{\partial B(0,1)}u_i(\nu)^2 d\nu + 2 \sum_{i=1}^D \sum_{j=1}^{i-1}\omega_{ij} \int_{\partial B(0,1)} u_{ij}(\nu)d\nu - 2 \sum_{i=1}^D \omega_i \int_{\partial B(0,1)} u_i(\nu)d\nu + \int_{\partial B(0,1)}d\nu
\end{align*}

Define the following quantities:

\begin{align*}
	C_1^{i} := \int_{\partial B(0,1)} u_i(\nu)^2d\nu\\
	C_2^{i} := \int_{\partial B(0,1)} u_i(\nu)d\nu\\
	C_3^{ij} := \int_{\partial B(0,1)} u_{ij}(\nu)d\nu\\
\end{align*}

Then the above is equivalent to:

\begin{align*}
	\mathcal{F}(\omega_1,...,\omega_D) &= \sum_{i=1}^D \omega_i^2 C_1^i + 2 \sum_{i=1}^D \sum_{j=1}^{i-1}\omega_{ij} C_3^{ij} - 2 \sum_{i=1}^D \omega_i C_2^i + \sim
\end{align*}

Where $\sim$ is a constant that depends only on the dimension $d$. Our goal is to find the minimum of $\mathcal{F}$ over all weights $\omega_1,...,\omega_D$. For this we will use Newton's Method for Optimization which requires us to calculate the gradient of $\mathcal{F}$ as well as the Hessian of $\mathcal{F}$. We'll start with the gradient. Let $k = 1,...,D$.

\begin{align*}
	\frac{\partial \mathcal{F}}{\partial \omega_k} &= \frac{\partial}{\partial \omega_k} \left( \sum_{i=1}^D \omega_i^2 C_1^i \right) + 2\frac{\partial}{\partial \omega_k} \left( \sum_{i=1}^D \sum_{j=1}^{i-1}\omega_{ij} C_3^{ij} \right) -2\frac{\partial}{\partial \omega_k} \left( \sum_{i=1}^D \omega_i C_2^i \right) + 0\\
	&= 2 \omega_k C_1^k + 2 \sum_{m=1, m \neq k}^{D}\omega_{m}C_{3}^{km} - 2 C_2^k\\
	&= 2 \omega_k C_1^k + 2 \sum_{m=1}^{D}\omega_{m}C_{3}^{km} - 2\omega_kC_3^{kk} - 2 C_2^k\\
\end{align*}

Next we have the Hessian matrix. Since $\mathcal{F}\in \mathcal{C}^2$, the second order partial derivatives commute by Clairaut's theorem. Then we have:

\begin{align*}
	\frac{\partial^2 \mathcal{F}}{\partial \omega_k \partial \omega_\ell} &= 2\frac{\partial}{\partial \omega_\ell} \left(\omega_k C_1^k\right) + 2 \frac{\partial}{\partial \omega_\ell} \left(  \sum_{m=1}^{D}\omega_{m}C_{3}^{km} \right) - 2\frac{\partial}{\partial \omega_\ell} \left( \omega_kC_3^{kk} \right) - 2 \frac{\partial}{\partial \omega_\ell} \left(C_2^k \right)\\
	&= \begin{cases}
		2C_3^{k\ell}, & \text{if $\ell < k$}\\
		2C_3^{k\ell} & \text{if $\ell > k$ }\\
		2C_1^k & \text{if $\ell = k$} 
	\end{cases}
\end{align*}

Next let's examine our constants a little closer, first in 2D. Suppose that $u_i = (x_i,y_i)$. Then:

\begin{align*}
C_1^i &= \int_{\partial B(0,1)} u_i(\nu)^2 d\nu\\
&= \int_0^{2\pi} ((x_i,y_i) \cdot \nu)^2 d\theta\\
&= \int_0^{2\pi} (x_i \cos \theta + y_i \sin \theta)^2 d\theta\\
&= x_i^2 \int_0^{2\pi} \cos^2(\theta)d\theta + 2x_iy_i \int_0^{2\pi} \cos(\theta) \sin(\theta)d\theta  + y_i \int_0^{2\pi} \sin^2(\theta) d\theta\\
&=  \pi(x_i^2+y_i^2)\\
&= \pi\|u_i\|^2 
\end{align*}

Alternatively, we may arrive at the same conclusion using symmetry. Note that we can always apply a coordinate transformation in the form of a rotation to assume that $u_i$ lies along the vertical axis. Normally with a coordinate transformations, we would have to scale our differential based on the determinant of the Jacobian of the transformation. However, rotations are linear transformations, so the determinant of the Jacobian of the transformation is simply the factor by which it scales areas - and rotations don't change areas at all so this factor is equal to 1 and we get to do it for free (another way to think about this is that rotations are always defined by unitary matrices which always have determinant 1). Then $u_i(\nu) = \|u_i \|\langle (0,1), (\cos \theta,\sin \theta)\rangle = \|u_i\| \sin(\theta)$ and $\|u_i\|^2 \int_0^{2\pi} \sin^2(\theta)d\theta = \pi\|u_i\|^2$.\\

We will apply the same symmetry to the next constant as well.

\begin{align*}
C_2^{i} &:= \int_{\partial B(0,1)} u_i(\nu)d\nu\\\\
&= \|u_i\| \int_0^{2\pi} |\sin(\theta)|d\theta\\
&= 2 \|u_i\| \int_0^\pi \sin(\theta)d\theta\\
&= 4 \|u_i\|
\end{align*}

And finally for the last constant:

\begin{align*}
C_3^{ij} &= \int_{\partial B(0,1)}u_{ij}(\nu)d\nu\\
&= \int_{\partial B(0,1)} |\langle u_i, \nu \rangle \cdot \langle u_j, \nu \rangle| d\nu\\
\end{align*}

In 3D, the derivation for $C_1$ and $C_2$ are similar, with the main change coming from working in spherical coordinates.

Here we have $\nu = (\sin \theta \cos \phi, \sin \theta \sin \phi, \cos \theta)$ where $\theta$ is measured down from the positive $z$ axis and $\phi$ is measured positively from the $x$ axis. We then have bounds for integrating over the unit sphere: $\phi \in [0,2\pi]$ and $\theta \in [0,\pi]$. Then the volume element is $\sin \theta d\theta d\phi$ and we can do two rotations so that $u_i = \| u_i \|(0,0,1)$ without loss of generality.\\

Then we have:

\begin{align*}
	&= \|u_i \| \int_0^{2\pi} \int_0^\pi |\cos \theta| \sin \theta d\theta d\phi\\
	&= 4 \pi \| u_i \| \int_0^{\frac{\pi}{2}}\cos \theta \sin \theta d\theta\\
	&= 2\pi \|u_i\| 
\end{align*}

For Simpsons rule:

\begin{align*}
\int_a^b f(x) dx &\approx \frac{b-a}{6}\left[f(a) + 4f\left(\frac{a+b}{2} +f(b)\right) \right]
\end{align*}

For the mixed $C_3^{ij}: f(x) = |\langle u_i, \nu \rangle \cdot \langle u_j, \nu \rangle|$\\

Points $A = (a_1,a_2)$ and $B = (b_1,b_2)$\\

$\vec{AB}:= (b_1-a_1,b_2-a_2)$

\section{Min-Cut Max-Flow and the $L^1TV$}

We can represent a graph as a two-dimensional characteristic function $f(x,y)$ which takes on the value $1$ if $x$ is connected to $y$ and the value $0$ if $x$ is not connected to $y$. 



\end{document}
